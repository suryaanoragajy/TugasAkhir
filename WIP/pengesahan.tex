\newpage
\thispagestyle{empty}
\newgeometry{
	top=30mm,left=20mm,right=30mm,bottom=25mm
}
\begin{center}
	\textbf{LEMBAR PENGESAHAN}\\
	\vskip30pt
	
	\textbf{PEMETAAN DAN PEMODELAN BAWAH PERMUKAAN POTENSI PANAS BUMI BERDASARKAN DATA LANDSAT 8 DAN METODE GRAVITASI}\\
	
	\vskip30pt
	\textbf{TUGAS AKHIR}\\
	\doublespacing
	Diajukan untuk memenuhi salah satu syarat\\
	memperoleh gelar Sarjana Sains\\
	Program Studi S-1 Fisika\\
	Departemen Fisika\\
	Fakultas Sains dan Analitika Data\\
	Institut Teknologi Sepuluh Nopember\\
	\vskip 20pt
	
	Oleh : \textbf{Surya Anoraga Justitia Yusman}\\
	NRP : 01111940000019\\
	
	\vskip 50pt
	Disetujui oleh Tim Penguji Proposal Tugas Akhir:
\end{center}
		\vskip 35pt
		Dr.rer.nat. Eko Minarto, M.Si \hfill Pembimbing \hfill (\dotfill)\\
		NIP. 19750205 199903 1 004
\begin{center}
	\vfill
	SURABAYA\\
	April, 2023
\end{center}

\restoregeometry
%\addcontentsline{toc}{chapter}{LEMBAR PENGESAHAN}
\newpage
\thispagestyle{empty}
\newgeometry{
	top=30mm,left=30mm,right=20mm,bottom=25mm
}

\begin{center}
	\textbf{ABSTRAK}\\
	\vskip 20pt
	\textbf{PEMETAAN DAN PEMODELAN BAWAH PERMUKAAN POTENSI PANAS BUMI BERDASARKAN DATA LANDSAT 8 DAN METODE GRAVITASI}
	\vskip 20pt
\end{center}

\begin{table}[h]
	\begin{tabular}{ll}
		Nama Mahasiswa / NRP &: Surya Anoraga Justitia Yusman / 01111940000019\\
		Departemen &: Fisika - FSAD\\
		Dosen Pembimbing &: Dr.rer.nat. Eko Minarto
	\end{tabular}
\end{table}

\vskip 10pt
\textbf{Abstrak}
\vskip 5pt
\hspace{25pt} Indonesia merupakan salah satu negara dengan potensi panas bumi terbesar di dunia. Potensi panas bumi yang telah teridentifikasi sebanyak 331 titik yang tersebar di 30 provinsi dengan kapasitas 11.073 MW berupa sumber daya dan 17.506 MW berupa potensi. Akan tetapi potensi yang besar ini baru dimanfaatkan sebesar 1.698,5 MW atau sekitar 9,3\% dari total cadangan panas bumi. Salah satu titik potensi panas bumi yang belum dimanfaatkan berlokasi di daerah Tiris-G.Lamongan, Probolinggo, Jawa Timur. Gunung Lamongan adalah salah satu gunung api aktif di Jawa Timur. Gunung ini bertipe strato dengan ketinggian mencapai 1671 mdpl. Secara administratif, G.Lamongan berada di dua kabupaten yaitu Probolinggo dan Lumajang. Di sisi timur G. Lamongan terdapat desa Segaran yang masuk dalam kecamatan Tiris. Di desa ini ditemukan dua mata air panas di lokasi yang berdekatan dengan suhu sekitar $42-46^{\circ}C$. Data geokimia menunjukkan adanya indikasi reservoir bertemperatur tinggi pada kedalaman yang signifikan, yang berasal dari Gunung Lamongan. Dalam menentukan potensi panas bumi perlu dilakukan survey awal yang dapat dilakukan dengan menggunakan metode-metode geofisika contohnya adalah metode penginderaan jauh dan metode gravitasi. Metode penginderaan jauh dapat dilakukan dengan menggunakan data citra satelit Landsat 8 untuk pemetaan sebaran potensi panas bumi dimana dapat ditunjukkan dengan adanya anomali \textit{land surface temperaature} (LST) pada suatu daerah serta dapat dikorelasikan dengan \textit{fault fracture density} (FFD) yang tinggi. Kemudian metode gravitasi dapat digunakan untuk interpretasi bawah permukaan struktur penyusun sistem panas bumi seperti \textit{caprock}, zona \textit{permeable} dan reservoir. Selain itu dengan menggunakan metode gravity dapat dilakukan analisis derivatif \textit{First Horizontal Derivative}(FHD) dan \textit{Second Vertical Derivative }(SVD) untuk menentukan lokasi dan jenis patahan.

\vskip 10pt
\textbf{Kata Kunci :} Analisis Derivatif, \textit{Fault Fracture Density}, \textit{Land Surface Temperature}, Potensi Panas Bumi 

\restoregeometry
%\addcontentsline{toc}{chapter}{ABSTRAK}
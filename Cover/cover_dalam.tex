% Sampul Dalam
\newpage
\thispagestyle{empty}
\newgeometry{
  top=30mm,right=30mm,left=20mm,bottom=25mm
}

% Kotak Biru
\begin{tikzpicture}[remember picture,overlay]
    \fill[white](current page.north west) rectangle ([yshift=-297mm]current page.north east);
    \fill[biru] (current page.north west) rectangle ([yshift=-50mm]current page.north east);
    \fill[white](current page.north west) rectangle ([yshift=-42.5mm]current page.north east);
\end{tikzpicture}

% Lambang ITS
\begin{tikzpicture}[remember picture, overlay]
    \node[anchor=south west, inner sep=0] at (-0.5,0.5) {\includegraphics[scale=0.4,width=2.6cm]{Figs/Lambang ITS.png}};
\end{tikzpicture}

% Isi
\vspace{8ex}
\fontsize{14pt}{14pt}{\fontfamily{lmss}\selectfont
{\textbf{TUGAS AKHIR - SF4801}}}\\

\vspace{11ex}

\doublespacing
\fontsize{18pt}{18pt}{\fontfamily{lmss}\selectfont
{\textbf{Pemetaan dan Pemodelan Bawah Permukaan Potensi \\ 
Panas Bumi Berdasarkan Data Landsat 8 dan Metode \\ 
Gravitasi}}}    

\vspace{15ex}
\fontsize{14pt}{14pt}{\fontfamily{cmss}\selectfont
{\textbf{Surya Anoraga Justitia Yusman}\\
NRP 01111940000019}}

\vspace{10ex}

\fontsize{14pt}{14pt}{\fontfamily{cmss}\selectfont
{Dosen Pembimbing\\
\textbf{Dr.rer.nat. Eko Minarto}\\
NIP 19750205 199903 1 004}}


\vspace{15ex}

\onehalfspacing
\fontsize{12pt}{14pt}{\fontfamily{cmss}\selectfont{
\textbf{DEPARTEMEN FISIKA}\\
Fakultas Sains dan Analitika Data \\
Institut Teknologi Sepuluh Nopember \\
Surabaya \\
2023}}

\restoregeometry

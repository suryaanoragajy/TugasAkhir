\chapter{PENDAHULUAN}
% === LATAR BELAKANG ===
\section{Latar Belakang}
\hspace{25pt}Produksi energi fosil dunia dalam beberapa tahun terakhir terus mengalami penurunan. Hal ini merupakan salah satu bentuk komitmen global dalam mengurangi emisi gas rumah kaca dunia\citep{OutlookEnergi2019}. Pemerintah Indonesia melalui PP No. 79 tahun 2014 tentang kebijakan energi nasional, menunjukkan perannya dalam mengurangi emisi gas rumah kaca dengan dengan menargetkan setidaknya bauran energi baru terbarukan (EBT) Indonesia mencapai 23\% pada tahun 2025 dan 31\% pada tahun 2050.

\hspace{25pt}Potensi EBT yang dimiliki Indonesia cukup besar dimana kapasitasnya mencapai 442 GW untuk pembangkit listrik dan 200 ribu Bph untuk bahan bakar sektor industri, transportasi, rumah tangga dan komersial\citep{OutlookEnergi2019}. Salah satu jenis potensi EBT tersebut adalah panas bumi dengan total kapasitas 28,5 GW. Potensi panas bumi yang telah teridentifikasi sebanyak 331 titik yang tersebar di 30 provinsi dengan kapasitas 11.073 MW berupa sumber daya dan 17.506 MW berupa potensi\citep{PanasBumiIndo2017}. Dengan ini Indonesia menjadi salah satu negara dengan potensi panas bumi terbesar di dunia\citep{esdm2018}. Hal tersebut berkat lokasi geografis Indonesia yang dilalui oleh tiga lempeng tektonik yaitu Indo-Australia, Eurasia dan Pasifik yang sekaligus membuat Indonesia berada dalam zona \textit{ring of fire} yang memiliki banyak gunung berapi aktif dimana gunung berapi memiliki kontribusi besar dalam terbentuknya panas bumi. Akan tetapi potensi yang besar ini baru dimanfaatkan sebesar 1.698,5 MW atau sekitar 9,3\% dari total cadangan panas bumi. Salah satu titik potensi panas bumi yang belum dimanfaatkan berlokasi di daerah Tiris-G.Lamongan, Probolinggo, Jawa Timur \citep{PanasBumiIndo2017}.

\hspace{25pt} Gunung Lamongan adalah salah satu gunung api aktif di Jawa Timur. Gunung ini bertipe strato dengan ketinggian mencapai 1671 mdpl. Secara administratif, G.Lamongan berada di dua kabupaten yaitu Probolinggo dan Lumajang. Di sisi timur G. Lamongan terdapat desa Segaran yang masuk dalam kecamatan Tiris. Di desa ini ditemukan dua mata air panas di lokasi yang berdekatan dengan suhu sekitar $42-46^{\circ}C$. Data geokimia menunjukkan adanya indikasi reservoir bertemperatur tinggi pada kedalaman yang signifikan, yang berasal dari Gunung Lamongan \citep{GLamongan}.

\hspace{25pt}Dalam menentukan potensi panas bumi perlu dilakukan survey awal yang dapat dilakukan menggunakan metode-metode geofisika contohnya adalah metode penginderaan jauh dan metode gravitasi. Metode penginderaan jauh dapat dilakukan dengan menggunakan data citra satelit Landsat 8 dimana dapat menunjukkan adanya anomali \textit{land surface temperaature} (LST) pada suatu daerah. Anomali ini terjadi pada suatu manifestasi panas bumi dan juga permukaan yang terdapat sesar atau rekahan dimana hal ini juga dapat dikorelasikan dengan \textit{fault fracture density} (FFD) yang tinggi. Maka dari itu metode ini dapat digunakan untuk pemetaan potensi panas bumi yang selanjutnya dapat dilakukan pemodelan bawah permukaan dengan menggunakan metode gravitasi sehingga dapat menginterpretasi struktur penyusun sistem panas bumi seperti \textit{caprock}, zona \textit{permeable} dan reservoir. Selain itu dengan menggunakan metode gravitasi dapat dilakukan analisis derivatif \textit{First Horizontal Derivative }(FHD) dan \textit{Second Vertical Derivative }(SVD) untuk menentukan lokasi dan jenis patahan.


% === RUMUSAN MASALAH ===
\section{Rumusan Masalah}
\hspace{22pt} Permasalahan yang muncul berdasarkan latar belakang di atas antara lain:
\begin{enumerate}
    \item Bagaimana sebaran potensi panas bumi di daerah Tiris-G. Lamongan?
    \item Bagaimana interpretasi bawah permukaan sistem panas bumi di daerah Tiris-G.Lamongan?\sloppy
    
\end{enumerate}


% === BATASAN MASALAH ===
\section{Batasan Masalah}
\hspace{22pt}Penelitian ini dilakukan dengan menimbang batasan masalah sebagai berikut:
\begin{enumerate}
    \item Penelitian menggunakan data sekunder Landsat 8 yang diperoleh dari \href{https://earthexplorer.usgs.gov/}{https://earthexplorer.usgs.gov/} serta data \textit{gravity} yang diperoleh dari \href{https://ddfe.curtin.edu.au/models/GGMplus/}{https://ddfe.curtin.edu.au/models/GGMplus/}\sloppy
\end{enumerate}


% === TUJUAN PENELITIAN ===
\section{Tujuan Penelitian}
\hspace{22pt}Berdasarkan permasalahan diatas, tujuan dari penelitian ini antara lain:
\begin{enumerate}
    \item Mengetahui sebaran potensi panas bumi di daerah Tiris-G. Lamongan
    \item Melakukan interpretasi bawah permukaan sistem panas bumi di daerah Tiris-G. Lamongan
\end{enumerate}

% === MANFAAT PENELITIAN ===
\section{Manfaat Penelitian}
\hspace{22pt}Manfaat dari penelitian ini adalah memberikan informasi terkait penggunaan data Landsat 8 yang dapat digunakan untuk mengetahui sebaran potensi panas bumi serta penggunaan metode gravitasi yang dapat digunakan untuk interpretasi bawah permukaan.

%\section{Sistematika Penulisan}